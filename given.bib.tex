# %% [markdown]
# Import necessary libraries

# %%
import pandas as pd
import matplotlib.pyplot as plt
import statsmodels.api as sm 

# %% [markdown]
# Load the dataset

# %%
data= pd.read_csv('ECONTRADE.csv')

# %% [markdown]
# Display initial data

# %%
print("Initial Data:")
print(data.head())

# %% [markdown]
# Check for missing values

# %%
print("\nMissing Values Before Cleaning:")
print(data.isnull().sum())


# %% [markdown]
# Drop rows with missing values

# %%
data_cleaned = data.dropna()

# %%
 Check if any missing values remain

# %%
print("\nMissing Values After Cleaning:")
print(data_cleaned.isnull().sum())

# %% [markdown]
#  Converting currency columns from string to float and removing commas

# %%
# Converting currency columns from string to float and removing commas
data_cleaned['Interprovincial exports of total products'] = data_cleaned['Interprovincial exports of total products'].replace(',', '', regex=True).astype(float)
data_cleaned['International exports of total products'] = data_cleaned['International exports of total products'].replace(',', '', regex=True).astype(float)


# %% [markdown]
# Creating Panel ID
# 

# %%
data_cleaned['Panel_ID'] = data_cleaned['ID'].astype(str) + "_" + data_cleaned['Year'].astype(str)


# %% [markdown]
#  Display cleaned data with Panel ID

# %%
print("\nCleaned Data with Panel IDs:")
print(data_cleaned.head())

# %% [markdown]
#  Basic data analysis - Descriptive statistics

# %%
print("\nDescriptive Statistics:")
print(data_cleaned.describe())

# %% [markdown]
#  Saving the cleaned data to a new CSV file

# %%
output_file_path = 'cleaned_data_with_panel_ids.csv'
data_cleaned.to_csv(output_file_path, index=False)
print(f"\nCleaned data saved to {output_file_path}")

# %% [markdown]
# Group data by 'Provinces' and 'Year' and sum the exports

# %%
grouped_data = data_cleaned.groupby(['Provinces', 'Year']).sum().reset_index()


# %% [markdown]
# Pivot data for easier plotting

# %%
pivot_interprovincial = grouped_data.pivot(index='Year', columns='Provinces', values='Interprovincial exports of total products')
pivot_international = grouped_data.pivot(index='Year', columns='Provinces', values='International exports of total products')

# %% [markdown]
# Plotting the trends

# %%
plt.figure(figsize=(14, 7))
plt.subplot(1, 2, 1)
pivot_interprovincial.plot(ax=plt.gca())
plt.title('Trend of Interprovincial Exports by Province')
plt.xlabel('Year')
plt.ylabel('Total Exports (in thousands)')

plt.subplot(1, 2, 2)
pivot_international.plot(ax=plt.gca())
plt.title('Trend of International Exports by Province')
plt.xlabel('Year')
plt.ylabel('Total Exports (in thousands)')

plt.tight_layout()
plt.show()

# %%
grouped_data.head()

# %%


# %%
# Group data by province and calculate sums of exports
province_interprovincial = data_cleaned.groupby('Provinces')['Interprovincial exports of total products'].sum().reset_index()
province_international = data_cleaned.groupby('Provinces')['International exports of total products'].sum().reset_index()

# Create a figure with two subplots
fig, axs = plt.subplots(1, 2, figsize=(24, 12))

# Create a pie chart for interprovincial exports
axs[0].pie(province_interprovincial['Interprovincial exports of total products'], labels=province_interprovincial['Provinces'], autopct='%1.1f%%', startangle=90, textprops={'fontsize': 12})
axs[0].set_title('Interprovincial Exports by Province', fontsize=18)
axs[0].axis('equal')

# Create a pie chart for international exports
axs[1].pie(province_international['International exports of total products'], labels=province_international['Provinces'], autopct='%1.1f%%', startangle=90, textprops={'fontsize': 12})
axs[1].set_title('International Exports by Province', fontsize=18)
axs[1].axis('equal')

# Layout so plots do not overlap
fig.tight_layout()

plt.show()

# %%



